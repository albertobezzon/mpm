% !TEX encoding = UTF-8
\documentclass[a4paper]{article}
\usepackage[utf8]{inputenc}
\usepackage[T1]{fontenc}
\usepackage{lmodern}
\usepackage{enumitem}
\PassOptionsToPackage{hyphens}{url}\usepackage{hyperref}
\usepackage[a4paper,top=2.5cm,bottom=2.5cm,left=2cm,right=2cm]{geometry}
%\usepackage{simplemargins}

%\usepackage[square]{natbib}
\usepackage{amsmath}
\usepackage{amsfonts}
\usepackage{amssymb}
\usepackage{graphicx}

%#######################################

\hypersetup{
	colorlinks=true,
	urlcolor=blue,
}

%#######################################

\begin{document}
\pagenumbering{gobble}

\Large
 \begin{center}
Il framework Flutter\\ 
\medskip

\end{center}

\begin{flushleft}
	\large
	Studenti: Alberto Bezzon e Tommaso Carraro\\
\end{flushleft}

\normalsize

Approccio scientifico (dati)
Stato dell'arte del FW (quanto è diffuso, chi ne parla SO e altre community, clienti)

\section*{Introduzione}
\textbf{Flutter} è un framework cross-platform free ed open-source creato da Google che consente la creazione di applicazioni native per iOS e Android con una singola codebase\footnote{Codebase: indica l'intera collezione di codice sorgente usata per costruire un'applicazione}. Il codice sorgente viene compilato direttamente in codice ARM, utilizzando la GPU e può avere accesso alle API e ai servizi della piattaforma.
Le tre principali caratteristiche che sono i punti forza di Flutter sono (da estendere nel dettaglio):
\begin{itemize}
	\item permette un rapido sviluppo, grazie a \textit{hot reload} dato che modificando il codice è possibile vedere in pochissimo tempo le modifiche apportate all'applicazione, senza perdere lo stato di quest'ultima. Inoltre, il bundle contiene dei widget personalizzabili nativi ed è possibile utilizzarlo con l'IDE preferito dallo sviluppatore;
	\item UI espressiva e flessibile;
	\item Performance native.
\end{itemize}

\section*{Architettura del framework}

\section*{Chi sta utilizzando Flutter?}



\section*{Fonti}
\begin{enumerate}[label={[\arabic*]}]
	\item Flutter. \url{https://flutter.dev/}
	\item Profiling Flutter Applications Using the Timeline.\\ \url{https://medium.com/flutter-io/profiling-flutter-applications-using-the-timeline-a1a434964af3}
	\item React Native vs Flutter, cross-platform mobile application frameworks. \\ \url{https://www.theseus.fi/bitstream/handle/10024/146232/thesis.pdf?sequence=1\&isAllowed=y}
	\item How fast is Flutter? I built a stopwatch app to find out. \url{https://medium.freecodecamp.org/how-fast-is-flutter-i-built-a-stopwatch-app-to-find-out-9956fa0e40bd}
	\item Youtube's video.
\end{enumerate}

\end{document}