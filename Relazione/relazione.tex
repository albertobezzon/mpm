\documentclass[12pt, a4paper, titlepage]{report}
\usepackage[a4paper,top=2.5cm,bottom=2cm,left=2cm,right=2cm]{geometry}
\usepackage[italian]{babel}
\usepackage[hidelinks]{hyperref}
\usepackage{graphicx}
\usepackage{algorithm}
\usepackage{capt-of}
\usepackage{amsmath}
\usepackage{longtable}
\usepackage{caption}
\usepackage{booktabs}
\usepackage{algpseudocode}
\usepackage{makecell}
\usepackage[utf8]{inputenc}
\usepackage{fixltx2e}
\usepackage{listings}
\usepackage{hyperref}
\usepackage{makecell}
\usepackage{changepage}
\usepackage{paracol}
\usepackage{siunitx}
\usepackage{lipsum}
\usepackage{float}
\usepackage{subfig}

\usepackage[swapnames]{frontespizio}

\usepackage{fancyhdr}
\usepackage{lastpage}
%\pagestyle{fancy}
%\fancyhf{}
%\cfoot{Pagina \thepage\ di \pageref{LastPage}}

\setcounter{tocdepth}{6}
\setcounter{secnumdepth}{6}
\newcommand\tab[1][1cm]{\hspace*{#1}}
\renewcommand{\thesection}{\arabic{section}}%
\newcommand{\myparagraph}[1]{\paragraph{#1}\mbox{} \mbox{}}
\newcommand{\tommaso}{Tommaso Carraro}
\newcommand{\alberto}{Alberto Bezzon}
	
\begin{document}
	
	\begin{frontespizio}
		\begin{Preambolo*}
			\usepackage{fourier}
		\end{Preambolo*}
		\Universita{Padova}
		\Dipartimento{Matematica}
		\Scuola{Corso di Mobile Programming and Multimedia}
		\Filigrana [height=7.5cm,before=0.28,after=1]{logo-unipd.png}
		\Titolo{\vspace{8cm}\Huge{Relazione del progetto SmartOrder}}
		\Relatore{Alberto Bezzon 1211016\\Tommaso Carraro }
		\NRelatore{Studenti}{}
		\Annoaccademico{2018-2019}
	\end{frontespizio}
	
	%\maketitle
	%\pagestyle{empty}
	\setcounter{page}{2}
	\tableofcontents
	\newpage
	\listoffigures
%	\newpage
%	\listoftables
%	\setcounter{table}{0}
	\newpage	    
	\renewcommand*{\arraystretch}{2}
	\pagestyle{fancy}
	\fancyhf{}
	\rhead{Relazione del progetto SmartOrder}
	\lhead{Corso di Mobile Programming and Multimedia \\ Università di Padova}
	\cfoot{\thepage}
	%\setlength{\headsep}{2cm}
	
	\section{Introduzione}
	Il progetto consiste nello sviluppo di un'applicazione, chiamata SmartOrder, per effettuare ordini dal proprio fornitore. L'applicazione è stata sviluppata per smartphone utilizzando il framework cross-platform PhoneGap.
	
	\section{Il framework cross-platform adottato}
	Il framework cross-platform scelto per lo sviluppo dell'applicazione è PhoneGap. La scelta è ricaduta su questo framework perché un membro del gruppo aveva già familiarità con questo framework e, inoltre, per realizzare un'applicazione con PhoneGap è richiesta la conoscenza di HTML, CSS e Javascript; in questi linguaggi la nostra competenza è già ottima, questo ha facilitato di molto il lavoro rispetto ad imparare un nuovo linguaggio.
	
	\section{Tutorial}
	Le pagine realizzate sono le seguenti:
	\begin{itemize}
		\item login.html;
		\item homepage.html;
		\item order.html;
		\item article.html;
		\item articles.html;
		\item newpage.html;
		\item inventory.html;
	\end{itemize}
	Ad ognuna di queste pagine è associato un foglio di stile e un file Javascript; inoltre sono presenti un file CSS e un file Javascript generali.
	
	\subsection{Login e tutorial}
	
	L'applicazione, una volta installata, è utilizzabile. All'avvio viene mostrata la pagina di login per accedere al servizio. Qui è necessario inserire le credenziali concesse dal proprio fornitore; inoltre, è presente una checkbox "Resta collegato" per evitare che ad ogni apertura dell'applicazione venga richiesto il login. In seguito al login, soltanto al primo avvio, viene mostrato un breve tutorial sul funzionamento dell'applicazione. In fondo al tutorial è presente un bottone che permette di iniziare ad utilizzare l'app. Il tutorial non è obbligatorio ed è disponibile nel menu dell'applicazione, nel caso in cui un utente si trovasse in difficoltà. Completato il tutorial si viene reindirizzati al \textit{Carrello}.
	
	\subsection{Carrello}
	La pagina Carrello è la pagina principale dell'applicazione, essa ad ogni nuovo ordine sarà vuota e visualizzerà ogni articolo che si deciderà di acquistare, con la relativa quantità, prezzo e prezzo parziale (prezzo unitario x quantità). Ad ogni articolo di questa pagina sono associati due bottoni: 
	\begin{itemize}
		\item \textbf{MODIFICA:} è il bottone che permette di modificare la quantità di un articolo. Una volta effettuata la modifica è necessario confermarla per apportarla definitivamente, premendo sul bottone "Conferma", oppure annullarla, tramite il bottone "Annulla";
		\item \textbf{ELIMINA:} è il bottone che rimuove l'articolo dal carrello. Se tale bottone viene premuto, verrà chiesta conferma prima di effettuare l'operazione. (La spiegazione del perché viene rimandata al capitolo sul mobile design).
	\end{itemize}
	In questa pagina, è presente il prezzo totale di tutti gli articoli inseriti nel carrello.
	Una volta inseriti nel carrello tutti gli articoli che si desidera acquistare, è possibile completare l'ordine, premendo sul bottone "Invia" in basso. Altrimenti è possibile svuotare il Carrello, cioè eliminare tutti gli articoli presenti in esso.
	
	\subsubsection{Aggiungere un articolo al carrello}
	E' possibile aggiungere un articolo al carrello in due modi: 
	\begin{enumerate}
		\item Tramite scansione del codice a barre mediante il bottone "Scan" nella pagina Carrello;
		\item Dalla pagine Inventario, selezionando l'articolo che si desidera aggiungere e inserendo la quantità. Una volta immessa la quantità desiderata, è necessario premere su "Conferma" affinché l'articolo venga inserito nel carrello oppure premere "Annulla" per annullare l'inserimento del nuovo articolo. Nella pagina Inventario è possibile ricercare un'articolo digitando la chiave di ricerca nella textbox in alto.
	\end{enumerate}
	In entrambi i casi, una volta selezionato l'articolo vengono visualizzate le sue informazioni (nome e descrizione) e nel riquadro in basso è presente il prezzo di quell'articolo per la quantità inserita.
	\\ \textbf{Attenzione:} nel caso in cui un articolo si cercasse di aggiungere al carrello un articolo già presente, il sistema mostrerà un messaggio nel quale chiederà se si vuole modificare la quantità relativa all'articolo.
	
	\subsection{Ordini}
	Dal menu dell'applicazione è possibile accedere alla lista di tutti gli ordini effettuati tramite la voce di menu "Ordini". Premendo su tale voce si verrà reindirizzati alla pagina Ordini, dove per ogni ordine, vengono mostrate le informazioni più importanti. E' possibile consultare tutti gli articoli acquistati in un determinato ordine, tramite il bottone "DETTAGLI".
	\noindent In questa pagina è presente la possibilità di modificare la visualizzazione degli ordini, ordinandoli per data, dal più recente al meno recente, oppure per prezzo, dall'ordine con prezzo più basso all'ordine con prezzo più alto. Per selezionare il tipo di ordinamento è presente una select in fondo alla pagina.

	
	\subsection{Informazioni utili all'utilizzo}
	Per poter utilizzare l'applicazione è necessario essere connessi ad Internet, in quanto l'applicazione scarica i dati da un server. Abbiamo notato che con la rete Eduroam e Studenti.math.unipd.it dell'università, l'applicazione non funziona. Inoltre la mail deve essere certificata da Amazon 
	
	\section{Mobile design}
	In questa sezione vengono descritte le scelte implementative effettuate ai fini di una corretta progettazione dell'interfaccia dell'applicazione. Questa sezione è suddivisa in varie sottosezioni, dove ognuna descrive un'elemento dell'interfaccia e le motivazioni relative alla sua progettazione.
	
	Quasi tutte le pagine dell'applicazione sono progettate nel seguente modo:
	\begin{itemize}
		\item nella parte superiore è presente un banner che contiene il titolo della pagina e alla sinistra di quest'ultimo, il bottone del menu;
		\item la parte centrale della pagina contiene le informazioni;
		\item nella parte inferiore è presente un footer.
	\end{itemize}
	
	\subsection{Bottoni}
	\begin{figure}[H] 
		\centering
		\subfloat[Bottoni presenti nel footer della pagina carrello ]{\includegraphics[width=0.4\columnwidth]{img/buttons-cart}}
		\hspace*{2cm}
		\subfloat[Bottoni presenti nel footer delle apgine aggiungi e modifica articolo]{\includegraphics[width=0.4\columnwidth]{img/buttons-add}}
		\caption{Bottoni nel footer}
		\label{fig:buttons}
	\end{figure}
	Nelle pagine Carrello, Modifica o Aggiungi sono presenti dei bottoni nella parte bassa dello schermo (Figura \ref{fig:buttons}). L'applicazione richiede una frequente modifica dei dati quindi i controlli sono stati posizionati in una zona semplice da raggiungere e, seguendo la regola "Content always on top", si è preferito inserire questi bottoni in basso, infrangendo così la regola pratica che su Android i controlli devono essere posizionati nella parte alta dello schermo. 
	\begin{figure}[H] 
		\centering
		\subfloat[Bottoni nella pagina Carrello ]{\includegraphics[width=0.3\columnwidth]{img/item-cart}}
		\hspace*{1cm}
		\subfloat[Bottoni nella pagina Ordini]{\includegraphics[width=0.3\columnwidth]{img/item-order}}
		\hspace*{1cm}
		\subfloat[Bottoni nella pagina Inventario]{\includegraphics[width=0.3\columnwidth]{img/item-inventory}}
		\caption{Bottoni nella parte centrale delle pagine}
		\label{fig:item-buttons}
	\end{figure}
	\noindent I bottoni presenti in Figura \ref{fig:item-buttons} sono quelli presenti nella parte centrale dello schermo; essendo disposti nella confort zone sono facilmente raggiungibili e più piccoli rispetto ai bottoni del footer, in quanto si è seguita l'immagine in Figura \ref{fig:dimvspos}.
	\begin{figure}[H] 
		\centering
		\includegraphics[width=0.3\textwidth]{img/dimvspos}
		\caption{Dimensione vs. posizione dei bottoni}
		\label{fig:dimvspos}
	\end{figure}
	\noindent Nonostante la dimensione ridotta, sono comunque sufficientemente distanti da evitare un tap errato su un altro bottone.
	
	\subsection{Menu}
	\begin{figure}[H] 
		\centering
		\includegraphics[width=0.3\textwidth]{img/menu}
		\caption{Menu dell'applicazione}
		\label{fig:menu}
	\end{figure}
	Il menu dell'applicazione è rappresentato in Figura \ref{fig:menu}. Per aprirlo è presente il classico bottone del menu ad "hamburger" sulla barra in alto a sinistra. Il menu è suddiviso in due parti: 
	\begin{itemize}
		\item Informazioni relative all'utente che ha effettuato il login nell'applicazione, in particolare vengono visualizzati il nome dell'utente e il nome dell'azienda. Queste due informazioni non sono modificabili e pertanto sono disposte nella parte alta del menu dato che non è possibile interagire con esse;
		\item L'insieme di link che permettono di navigare tra le pagine dell'applicazione, in particolare sono disponibili i link al carrello, alla lista degli ordini, all'inventario. Inoltre, è possibile effettuare il logout dall'applicazione e visionare nuovamente il tutorial. Dato che è possibile interagire con essi, sono stati posizionati nella parte centrale del menu; in questo modo quando il menu compare questi bottoni ricadono nella confort zone.
	\end{itemize}

	\subsection{Pagina ordini}
	In generale, è una buona regola non affollare le interfacce. In particolare nella pagina ordini le informazioni sui vari ordini sono disponibili facilmente, mentre i dettagli di ogni ordine sono rimandati ad un'interazione successiva (progressive disclosure). Questo approccio favorisce la chiarezza sulla quantità di informazioni fornite ed evita l'affollamento dell'interfaccia.
	
	\subsection{Tastiera}
	L'utilizzo della tastiera è necessario in tre casi:
	\begin{itemize}
		\item Per inserire le credenziali necessarie per il login all'applicazione;
		\item Per effettuare la ricerca di un articolo;
		\item Per inserire la quantità del prodotto che si vuole inserire nel carrello.
	\end{itemize}
	Nel caso della quantità sono stati messi a disposizione due pulsanti (+/-) che, rispettivamente, aumentano e diminuiscono la quantità. Inoltre, se si desidera inserire un valore, premendo sulla textbox compare una tastiera con soli numeri in modo da aumentare la velocità dell'utente a digitare.
	
	\subsection{Tutorial}
	La prima volta che l'utente accede all'applicazione, dopo aver effettuato il login, viene visualizzato il tutorial per utilizzare l'applicazione. Il tutorial fornisce le istruzioni più importanti per l'utilizzo dell'applicazione, in quanto l'applicazione è abbastanza intuitiva. Alla fine del tutorial è presente un bottone che permette di iniziare ad utilizzare l'applicazione. Il tutorial comunque sarà disponibile dal menu, nel caso l'utente non ricordasse come utilizzare l'app.
	
	\subsection{Elementi nativi}
	Gli elementi nativi dell'applicazione sono la fotocamera e i dialog.
	
	Riteniamo che l'applicazione soddisfi i primi tre livelli della piramide di Maslow rimappata sui bisogni degli utenti, ovvero sia funzionale, affidabile e usabile.
	
	\section{Backend}
	Per il backend abbiamo deciso di appoggiarci ad AWS, dove sono ospitati i nostri server. In particolare, abbiamo utilizzato:
	\begin{itemize}
		\item Un server azure;
		\item un server contenente il database, per cui abbiamo utilizzato SQLserver;
	\end{itemize}
	
	\section{Conclusioni}
	Siamo soddisfatti dell'applicazione realizzata. Siamo consapevoli che la grafica può essere resa più accattivante per lo scopo dell'applicazione. La grafica poteva essere più semplice da realizzare se utilizzavamo un altro framework, come ad esempio React. 
	Il framework cross-platform scelto è bastato per realizzare un'applicazione funzionale e decente. Inoltre, per un'applicazione semplice come SmartOrder permette di ottenere delle buone performance.
\end{document}