\documentclass[12pt, a4paper, titlepage]{report}
\usepackage[a4paper,top=2.5cm,bottom=2cm,left=2cm,right=2cm]{geometry}
\usepackage[italian]{babel}
\usepackage[hidelinks]{hyperref}
\usepackage{graphicx}
\usepackage{algorithm}
\usepackage{capt-of}
\usepackage{amsmath}
\usepackage{longtable}
\usepackage{caption}
\usepackage{booktabs}
\usepackage{algpseudocode}
\usepackage{makecell}
\usepackage[utf8]{inputenc}
\usepackage{fixltx2e}
\usepackage{listings}
\usepackage{hyperref}
\usepackage{makecell}
\usepackage{changepage}
\usepackage{paracol}
\usepackage{siunitx}
\usepackage{lipsum}

\usepackage[swapnames]{frontespizio}

\usepackage{fancyhdr}
\usepackage{lastpage}
%\pagestyle{fancy}
%\fancyhf{}
%\cfoot{Pagina \thepage\ di \pageref{LastPage}}

\setcounter{tocdepth}{6}
\setcounter{secnumdepth}{6}
\newcommand\tab[1][1cm]{\hspace*{#1}}
\renewcommand{\thesection}{\arabic{section}}%
\newcommand{\myparagraph}[1]{\paragraph{#1}\mbox{} \mbox{}}
\newcommand{\tommaso}{Tommaso Carraro}
\newcommand{\alberto}{Alberto Bezzon}
	
\begin{document}
	
	\begin{frontespizio}
		\begin{Preambolo*}
			\usepackage{fourier}
		\end{Preambolo*}
		\Universita{Padova}
		\Dipartimento{Matematica}
		\Scuola{Corso di Mobile Programming and Multimedia}
		\Filigrana [height=7.5cm,before=0.28,after=1]{logo-unipd.png}
		\Titolo{\vspace{8cm}\Huge{Relazione del progetto SmartOrder}}
		\Relatore{Alberto Bezzon 1211016\\Tommaso Carraro }
		\NRelatore{Studenti}{}
		\Annoaccademico{2018-2019}
	\end{frontespizio}
	
	%\maketitle
	%\pagestyle{empty}
	\setcounter{page}{2}
	\tableofcontents
	\newpage
	\listoffigures
	\newpage
	\listoftables
	\setcounter{table}{0}
	\newpage	    
	\renewcommand*{\arraystretch}{2}
	\pagestyle{fancy}
	\fancyhf{}
	\rhead{Relazione del progetto SmartOrder}
	\lhead{Corso di Mobile Programming and Multimedia \\ Università di Padova}
	\cfoot{\thepage}
	%\setlength{\headsep}{2cm}
	
	\section{Introduzione}
	Il progetto consiste nello sviluppo di un'applicazione, chiamata SmartOrder, per effettuare ordini dal proprio fornitore. L'applicazione è stata sviluppata utilizzando il framework cross-platform PhoneGap.
	
	\section{Il framework cross-platform adottato}
	Il framework cross-platform scelto per lo sviluppo dell'applicazione è PhoneGap. La scelta è ricaduta su questo framework perché un'applicazione sviluppata con PhoneGap è un'applicazione web a cui si aggiunge il motore di rendering webkit. Inoltre, per un'applicazione semplice come SmartOrder permette di ottenere delle buone performance.
	
	\section{Tutorial}
	
	\subsection{Informazioni utili all'utilizzo}
	Per poter utilizzare l'applicazione è necessario essere connessi ad Internet, in quanto l'applicazione scarica i dati da un server. Abbiamo notato che con la rete Eduroam e Studenti.math.unipd.it dell'università, l'applicazione non funziona.
	
	\section{Mobile design}
	
	\subsection{Elementi nativi}
	
	\section{Backend}
	
	\section{Conclusioni}
\end{document}